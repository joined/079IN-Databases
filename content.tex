% Non compilare questo file direttamente, compilare main.tex
\frame{\titlepage}

\begin{frame}{Analisi dei requisiti}
    \begin{itemize}
        \item Requisiti espressi in linguaggio naturale
        \item Glossario dei termini
        \item Strutturazione dei requisiti
    \end{itemize}
\end{frame}

\begin{frame}[t]{Requisiti espressi in linguaggio naturale}
    \begin{shadequote}[r]{}
        \small
        Si vuole realizzare una base di dati per gestire la rete delle biblioteche della regione Friuli Venezia-Giulia.
        
        Ci sono diverse biblioteche in tutte le province della regione, ed esse possono fare parte di diversi gruppi (e.g. Biblioteche Universitarie, Biblioteche Scolastiche).
        
        Le biblioteche gestiscono libri cartacei; lo stesso libro, identificato dal codice ISBN, si può trovare in diverse biblioteche e nella stessa biblioteca ci possono essere più copie dello stesso. Del libro si vogliono memorizzare anche il nome e l'anno.
        
        Ogni libro ha una lista di autori e fa parte di una o più categorie. Gli autori sono caratterizzati da nome e cognome, mentre le categorie hanno solo un nome.
    \end{shadequote}
\end{frame}

\begin{frame}[t]{Requisiti espressi in linguaggio naturale}
    \begin{shadequote}[r]{}
        \small
        Per richiedere un prestito le persone si possono registrare come membri presso una qualsiasi biblioteca fornendo nome, cognome e data di nascita. Bisogna essere maggiorenni per registrarsi. 

        I prestiti durano 30 giorni, e non è possibile richiedere in prestito un libro il giorno stesso che lo si è restituito. Ogni membro può prendere in prestito al massimo 5 libri alla volta. Dei membri si vuole memorizzare inoltre lo stato dell'iscrizione (attiva, sospesa) e il numero di ammonizioni.

        Se si riconsegna un libro in ritardo, si viene ammoniti. Alla quarta ammonizione, l'iscrizione viene sospesa per 30 giorni.

        Si vuole inoltre rendere disponibile la lista dei 10 libri più noleggiati in assoluto ad un utente esterno.
    \end{shadequote}
\end{frame}

\begin{frame}{Glossario dei termini}
    \scriptsize
    \def\arraystretch{1.4}
    \begin{center}
        \begin{tabular}{|c|m{5.5cm}|m{1.2cm}|m{1.5cm}|}
            \hline
            \textbf{Termine}    & \textbf{Descrizione} & \textbf{Sinonimi} & \textbf{Collegamenti logici} \\ \hline
            Libro               & Libro come entità astratta, che può essere presente in diverse biblioteche delle regione, anche in più copie &  & Copia libro, Autore, Categoria \\ \hline
            Autore              & Persona che ha scritto un libro, anche insieme ad altri. Ogni libro ne ha almeno uno & Scrittore & Libro \\ \hline
            Categoria           & Categoria di un libro. Una categoria può avere molti libri e un libro può esser in molte cat. & Tipologia   & Libro \\ \hline
            Copia libro         & Copia fisica di un libro presente in una biblioteca, data in prestito ai membri &  & Libro, Membro, Biblioteca  \\ \hline
            Membro              & Persona iscritta che può richiedere libri in prestito & Iscritto & Indirizzo, Prestito \\ \hline
        \end{tabular}
    \end{center}
\end{frame}

\begin{frame}{Glossario dei termini}
    \scriptsize
    \def\arraystretch{1.5}
    \begin{center}
        \begin{tabular}{|c|m{5.5cm}|m{1.2cm}|m{1.5cm}|}
            \hline
            \textbf{Termine}    & \textbf{Descrizione} & \textbf{Sinonimi} & \textbf{Collegamenti logici} \\ \hline
            Biblioteca          & Biblioteca situata presso un certo indirizzo &  & Indirizzo, Copia libro, Gruppo\\ \hline
            Gruppo              & Insieme di biblioteca accomunate da una certa caratteristica &  & Biblioteca \\ \hline
            Indirizzo           & Indicazione di posizione geografica di un luogo &  & Membro, Biblioteca \\ \hline
            Comune              & Ente territoriale di base & & Indirizzo, Provincia \\ \hline
            Provincia           & Ente locale con competenza su un gruppo di comuni & & Comune \\ \hline
            Prestito            & Noleggio da parte di un membro di una copia di un libro presso una biblioteca & Noleggio & Membro, Copia libro \\ \hline
        \end{tabular}
    \end{center}
\end{frame}

\begin{frame}{Strutturazione dei requisiti}
    \begin{itemize}
        \item Frasi di carattere generale:
            \begin{itemize}
                \item Si vuole realizzare una base di dati per gestire la rete delle biblioteche della regione Friuli Venezia-Giulia
            \end{itemize}
        \item Frasi relative alle biblioteche e ai gruppi:
            \begin{itemize}
                \item Ci sono diverse biblioteche in tutte le province della regione, ed esse possono fare parte di diversi gruppi (e.g. Biblioteche Universitarie, Biblioteche Scolastiche)
            \end{itemize}
        \item Frasi relative ai libri e alle copie:
            \begin{itemize}
                \item Le biblioteche gestiscono libri cartacei; lo stesso libro, identificato dal codice ISBN, si può trovare in diverse biblioteche e nella stessa biblioteca ci possono essere più copie dello stesso. Del libro si vogliono memorizzare anche il nome e l'anno
            \end{itemize}
        \item Frasi relative agli autori e alle categorie:
            \begin{itemize}
                \item Ogni libro ha una lista di autori e fa parte di una o più categorie. Gli autori sono caratterizzati da nome e cognome, mentre le categorie hanno solo un nome
            \end{itemize}
    \end{itemize}
\end{frame}

\begin{frame}{Strutturazione dei requisiti}
    \begin{itemize}
        \item Frasi relative ai membri:
            \begin{itemize}
                \item Per richiedere un prestito le persone si possono registrare come membri presso una qualsiasi biblioteca fornendo nome, cognome e data di nascita. Bisogna essere maggiorenni per registrarsi
            \end{itemize}
        \item Frasi relative ai prestiti:
            \begin{itemize}
                \item I prestiti durano 30 giorni, e non è possibile richiedere in prestito un libro il giorno stesso che lo si è restituito. Ogni membro può prendere in prestito al massimo 5 libri alla volta. Dei membri si vuole memorizzare inoltre lo stato dell'iscrizione (attiva, sospesa) e il numero di ammonizioni

                \item Se si riconsegna un libro in ritardo, si viene ammoniti. Alla quarta ammonizione, l'iscrizione viene sospesa per 30 giorni
            \end{itemize}
        \item Frasi aggiuntive:
            \begin{itemize}
                \item Si vuole inoltre rendere disponibile la lista dei 10 libri più noleggiati in assoluto ad un utente esterno
            \end{itemize}
    \end{itemize}
\end{frame}

\begin{frame}{Progettazione concettuale}
    \begin{itemize}
        \item Analisi delle entità e delle relazioni
        \item Modello ER
        \item Vincoli non esprimibili
    \end{itemize}
\end{frame}

\begin{frame}{Analisi delle entità e delle relazioni}
    È stata scelta la metodologia di progetto \emph{bottom-up}. \\

    Questa metodologia consiste nello sviluppare singolarmente in dettaglio le diverse parti del database, per poi connetterle. \\

    Vengono quindi presentate di seguito le parti che compongono il modello ER, analizzandone gli attributi insieme alla cardinalità delle relazioni.
\end{frame}

\begin{frame}{Analisi modello Libro/Autore/Categoria}
    \begin{itemize}
        \item Analisi delle entità
            \begin{itemize}
                \item \textit{Libro}
                    \begin{itemize}
                        \item Nome: il nome di copertina del libro
                        \item ISBN: il codice ISBN del libro
                        \item Anno: anno di pubblicazione del libro
                    \end{itemize}
                \item \textit{Categoria}
                    \begin{itemize}
                        \item Nome: il nome della categoria
                    \end{itemize}
                \item \textit{Autore}
                    \begin{itemize}
                        \item Nome: il nome dell'autore
                        \item Cognome: il cognome dell'autore
                    \end{itemize}
            \end{itemize}
    \end{itemize}
\end{frame}

\begin{frame}{Analisi modello Libro/Autore/Categoria}
    \begin{itemize}
        \item Analisi delle relazioni
            \begin{itemize}
                \item \textit{Appartenenza} (Libro - Categoria)
                    \begin{itemize}
                        \item Un libro appartiene ad una o più categorie
                        \item Una categoria contiene zero (categoria vuota) o più libri
                        \item Relazione $(1,N) \leftrightarrow (0,N)$
                    \end{itemize}
                \item \textit{Scrittura} (Libro - Autore)
                    \begin{itemize}
                        \item Un libro è scritto da uno o più autori
                        \item Un autore ha scritto zero (nessuno) o più libri
                        \item Relazione $(1,N) \leftrightarrow (0,N)$
                    \end{itemize}
            \end{itemize}
    \end{itemize}
\end{frame}

\begin{frame}{Modello ER Libro/Autore/Categoria}
    \begin{center}
        \begin{tikzpicture}[node distance=7em, every node/.style={scale=0.75}]
            % entità Libro
            \node[entity] (libro) {Libro};
            % relazione Libro - Autore
            \node[relationship, xshift=1cm] (scrittura) [right of=libro] {Scrittura} edge node[above]{(1,N)} (libro);
            % entità Autore
            \node[entity] (autore) [below of=scrittura] {Autore} edge node[left]{(0,N)} (scrittura);
            % relazione Libro - Categoria
            \node[relationship, xshift=-1cm] (appartenenza) [left of=libro] {\small Appartenenza} edge node[above]{(1,N)} (libro);
            % entità Categoria
            \node[entity] (categoria) [below of=appartenenza] {Categoria} edge node[right]{(0,N)} (appartenenza);
            % attributi Libro
            \node[attribute] (isbn) [above of=libro] {ISBN} edge (libro);
            \node[attribute] (nome_libro) [above left of=libro] {Nome} edge (libro);
            \node[attribute] (anno_pubblicazione) [above right of=libro] {Anno} edge (libro);
            % attributi Categoria
            \node[attribute] (nome_categoria) [left of=categoria] {Nome} edge (categoria);
            % attributi Autore
            \node[attribute] (nome_autore) [left of=autore] {Nome} edge (autore);
            \node[attribute] (cognome_autore) [right of=autore] {Cognome} edge (autore);
        \end{tikzpicture}
    \end{center}
\end{frame}

\begin{frame}{Analisi modello Indirizzo/Comune/Provincia}
    \begin{itemize}
        \item Analisi delle entità
            \begin{itemize}
                \item \textit{Indirizzo}
                    \begin{itemize}
                        \item Nome via: il nome della via privo di numero civico
                        \item Numero civico: il numero civico della via
                        \item Interno (opzionale): l'interno in caso di appartamento
                    \end{itemize}
                \item \textit{Comune}
                    \begin{itemize}
                        \item Codice catastale: codice catastale del comune
                        \item Nome: il nome del comune
                    \end{itemize}
                \item \textit{Provincia}
                    \begin{itemize}
                        \item Sigla: sigla di due lettere della provincia
                        \item Nome: nome della provincia
                    \end{itemize}
            \end{itemize}
    \end{itemize}
\end{frame}

\begin{frame}{Analisi modello Indirizzo/Comune/Provincia}
    \begin{itemize}
        \item Analisi delle relazioni
            \begin{itemize}
                \item \textit{Appartenenza} (Indirizzo - Comune)
                    \begin{itemize}
                        \item Un indirizzo appartiene ad un solo comune
                        \item Ad un comune appartengono zero o più indirizzi
                        \item Relazione $(1,1) \leftrightarrow (0,N)$
                    \end{itemize}
                \item \textit{Appartenenza} (Comune - Provincia)
                    \begin{itemize}
                        \item Un comune appartiene ad una sola provincia
                        \item Ad una provincia appartengono uno o più comuni
                        \item Relazione $(1,1) \leftrightarrow (1,N)$
                    \end{itemize}
            \end{itemize}
    \end{itemize}
\end{frame}

\begin{frame}{Modello ER Indirizzo/Comune/Provincia}
    \begin{center}
        \begin{tikzpicture}[node distance=9em, every node/.style={scale=0.65}]
            % entità Indirizzo
            \node[entity] (indirizzo) {Indirizzo};
            % relazione Indirizzo - Comune
            \node[relationship] (ind_appartenenza) [right of=indirizzo] {\small Appartenenza} edge node[above,xshift=1]{(1,1)}(indirizzo);
            % entità Comune
            \node[entity] (comune) [right of=ind_appartenenza] {Comune} edge node[above,xshift=-1]{(0,N)} (ind_appartenenza);
            % relazione Comune - Provincia
            \node[relationship] (com_appartenenza) [right of=comune] {\small Appartenenza} edge node[above,xshift=1]{(1,1)} (comune);
            % entità Provincia
            \node[entity] (provincia) [right of=com_appartenenza] {Provincia} edge node[above,xshift=-1]{(1,N)} (com_appartenenza);
            % attributi Provincia
            \node[attribute] (sigla_prov) [above of=provincia] {Sigla} edge (provincia);
            \node[attribute] (nome_prov) [below of=provincia] {Nome} edge (provincia);
            % attributi Comune
            \node[attribute] (cod_catast) [above of=comune] {Codice catastale} edge (comune);
            \node[attribute] (nome_com) [below of=comune] {Nome} edge (comune);
            % attributi Indirizzo
            \node[attribute] (civico) [below of=indirizzo] {N. civico} edge (indirizzo);
            \node[attribute] (nome_via) [above of=indirizzo] {Nome via} edge (indirizzo);
            \node[attribute] (interno) [below right of=indirizzo] {Interno} edge[dashed] (indirizzo);
        \end{tikzpicture}
    \end{center}
\end{frame}

\begin{frame}{Analisi modello Biblioteca/Gruppo}
    \begin{columns}[c]
        \begin{column}[T]{.35\textwidth}
            \begin{itemize}
                \item Analisi delle entità
                    \begin{itemize}
                        \item \textit{Biblioteca}
                            \begin{itemize}
                                \item Nome: nome della biblioteca
                            \end{itemize}
                        \item \textit{Gruppo}
                            \begin{itemize}
                                \item Nome: nome del gruppo
                            \end{itemize}
                    \end{itemize}
            \end{itemize}
        \end{column}
        \begin{column}[T]{.65\textwidth}
            \begin{itemize}
                \item Analisi delle relazioni
                    \begin{itemize}
                        \item \textit{Ubicazione} (Biblioteca - Indirizzo)
                            \begin{itemize}
                                \item Una biblioteca è ubicata presso un solo indirizzo
                                \item Ad un indirizzo può o può non essere ubicata una biblioteca
                                \item Relazione $(1,1) \leftrightarrow (0,1)$
                            \end{itemize}
                        \item \textit{Appartenenza} (Biblioteca - Gruppo)
                            \begin{itemize}
                                \item Una biblioteca appartiene a zero o più gruppi
                                \item Ad un gruppo appartengono zero o più biblioteche
                                \item Relazione $(0,N) \leftrightarrow (0,N)$
                            \end{itemize}
                    \end{itemize}
            \end{itemize}
        \end{column}
    \end{columns}
\end{frame}

\begin{frame}{Modello ER Biblioteca/Gruppo}
    \begin{center}
        \begin{tikzpicture}[node distance=9em, every node/.style={scale=0.7}]
            % entità Biblioteca
            \node[entity] (biblioteca) {Biblioteca};
            % relazione Biblioteca - Indirizzo
            \node[relationship] (ubicazione) [above of=biblioteca] {Ubicazione} edge node[right]{(1,1)} (biblioteca);
            % relazione Biblioteca - Gruppo
            \node[relationship] (appartenenza) [right of=biblioteca] {\small Appartenenza} edge node[above,xshift=2]{(0,N)} (biblioteca);
            % entità Gruppo
            \node[entity] (gruppo) [right of=appartenenza] {Gruppo} edge node[above,xshift=-1]{(0,N)} (appartenenza);
            % entità Indirizzo
            \node[entity] (indirizzo) [above of=ubicazione] {Indirizzo} edge node[right]{(0,1)} (ubicazione);
            % attributi Gruppo
            \node[attribute] (nome_gruppo) [above of=gruppo] {Nome} edge (gruppo);
            % attributi Biblioteca
            \node[attribute] (nome_biblioteca) [left of=biblioteca] {Nome} edge (biblioteca);            
        \end{tikzpicture}
    \end{center}
\end{frame}

\begin{frame}{Analisi modello Membro}
    \begin{columns}[c]
        \begin{column}[T]{.45\textwidth}
            \begin{itemize}
                \item Analisi delle entità
                    \begin{itemize}
                        \item \textit{Membro}
                            \begin{itemize}
                                \item Data di nascita: data di nascita del membro
                                \item Nome: nome del membro
                                \item Cognome: cognome del membro
                                \item Ammonizioni: numero di ammonizioni del membro
                                \item Data iscrizione: data di iscrizione del membro
                                \item Stato iscrizione: stato dell'iscrizione del membro
                            \end{itemize}
                    \end{itemize}
            \end{itemize}
        \end{column}
        \begin{column}[T]{.5\textwidth}
            \begin{itemize}
                \item Analisi delle relazioni
                    \begin{itemize}
                        \item \textit{Residenza} (Membro - Indirizzo)
                            \begin{itemize}
                                \item Un membro risiede presso un solo indirizzo
                                \item Ad un indirizzo possono risiedere zero o più membri
                                \item Relazione $(1,1) \leftrightarrow (0,N)$
                            \end{itemize}
                    \end{itemize}
            \end{itemize}
        \end{column}
    \end{columns}
\end{frame}

\begin{frame}{Modello ER Membro}
    \begin{center}
        \begin{tikzpicture}[node distance=9em, every node/.style={scale=0.7}]
            % entità Membro
            \node[entity] (membro) {Membro};
            % relazione Membro - Indirizzo
            \node[relationship] (residenza) [right of=membro] {Residenza} edge node[above]{(1,1)} (membro);
            % entità Indirizzo
            \node[entity] (indirizzo) [right of=residenza] {Indirizzo} edge node[above]{(0,N)} (residenza);
            % attributi di Membro
            \node[attribute] (nome) [above of=membro] {Nome} edge (membro);
            \node[attribute] (cognome) [above right of=membro] {Cognome} edge (membro);
            \node[attribute] (datanascita) [above left of=membro] {Data di nascita} edge (membro);
            \node[attribute] (dataisc) [below left of=membro] {Data iscrizione} edge (membro);
            \node[attribute] (stato_iscrizione) [below of=membro] {Stato iscr.} edge (membro);
            \node[attribute] (ammonizioni) [below of=membro,yshift=1cm,xshift=2cm] {Ammonizioni} edge (membro);
        \end{tikzpicture}
    \end{center}
\end{frame}

\begin{frame}{Analisi modello Copia libro/Prestito}
    \begin{itemize}
        \item Analisi delle entità
            \begin{itemize}
                \item \textit{Copia libro}
                    \begin{itemize}
                        \item Data aggiunta: data di aggiunta della copia al catalogo della biblioteca
                    \end{itemize}
            \end{itemize}
        \item Analisi delle relazioni
            \begin{itemize}
                \item \textit{È copia di} (Copia libro - Libro)
                    \begin{itemize}
                        \item Una copia fisica è tale di uno e di un solo libro
                        \item Un libro ha zero o più copie fisiche
                        \item Relazione $(1,1) \leftrightarrow (0,N)$
                    \end{itemize}
                \item \textit{Presenza} (Copia libro - Biblioteca)
                    \begin{itemize}
                        \item Una copia fisica è presente in una ed una sola biblioteca
                        \item In una biblioteca sono presenti zero o più copie fisiche di libri
                        \item Relazione $(1,1) \leftrightarrow (0,N)$
                    \end{itemize}
                \item \textit{Prestito} (Copia libro - Membro)
                    \begin{itemize}
                        \item Una copia fisica può essere o non essere in prestito ad un membro
                        \item Un membro può avere in prestito zero o più copie fisiche di libri
                        \item Relazione $(0,1) \leftrightarrow (0,N)$
                        \item Attributi: data inizio, data fine, data restituzione
                    \end{itemize}
            \end{itemize}
    \end{itemize}
\end{frame}

\begin{frame}{Modello ER Copia libro/Prestito}
    \begin{center}
        \begin{tikzpicture}[node distance=10em, every node/.style={scale=0.5}]
            % entità Libro
            \node[entity] (libro) {Libro};
            % relazione Libro - Copia libro
            \node[relationship] (e_copia_di) [right of=libro] {È copia di} edge node[above]{(0,N)} (libro);
            % entità Copia libro
            \node[entity] (copia_libro) [right of=e_copia_di] {Copia libro} edge node[above,xshift=-5]{(1,1)} (e_copia_di);
            % relazione Prestito
            \node[relationship] (prestito) [right of=copia_libro] {Prestito} edge node[above]{(0,1)} (copia_libro);
            % entità Membro
            \node[entity] (membro) [right of=prestito] {Membro} edge node[above]{(0,N)} (prestito);
            % relazione Copia libro - Biblioteca
            \node[relationship] (presenza) [below of=copia_libro] {Presenza} edge node[below right]{(1,1)} (copia_libro);
            % entità Biblioteca
            \node[entity] (biblioteca) [below of=presenza] {Biblioteca} edge node[right]{(0,N)} (presenza);
            % attributi Copia libro
            \node[attribute] (data_aggiunta) [above left of=copia_libro] {Data aggiunta} edge (copia_libro);
            % attributi Prestito
            \node[attribute] (data_inizio) [above left of=prestito] {Data inizio} edge (prestito);
            \node[attribute] (data_restituzione) [above of=prestito] {Data restituzione} edge (prestito);
            \node[attribute] (data_fine) [above right of=prestito] {Data fine} edge (prestito);
        \end{tikzpicture}
    \end{center}
\end{frame}

\begin{frame}{Modello ER}
    \begin{center}
        \begin{tikzpicture}[node distance=9em, every node/.style={scale=0.3}]
            % entità Libro
            \node[entity] (libro) {Libro};
            % attributi Libro
            \node[attribute] (nome_libro) [below left of=libro,xshift=1cm] {Nome} edge (libro);
            \node[attribute] (anno) [below right of=libro,xshift=-1cm] {Anno} edge (libro);
            % relazione Libro - Autore
            \node[relationship, xshift=1cm] (scrittura) [right of=libro] {Scrittura} edge node[above]{(1,N)} (libro);
            % entità Autore
            \node[entity] (autore) [right of=scrittura] {Autore} edge node[above,xshift=-1]{(0,N)} (scrittura);
            % attributi Autore
            \node[attribute] (nome_autore) [below left of=autore,xshift=1cm] {Nome} edge (autore);
            \node[attribute] (cognome_autore) [below right of=autore,xshift=-1cm] {Cognome} edge (autore);
            % relazione Libro - Categoria
            \node[relationship, xshift=-1cm] (libro_appartenenza) [left of=libro] {\small Appartenenza} edge node[above]{(1,N)} (libro);
            % entità Categoria
            \node[entity] (categoria) [left of=libro_appartenenza] {Categoria} edge node[above,xshift=2]{(0,N)} (libro_appartenenza);
            % attributi Categoria
            \node[attribute] (nome_categoria) [below of=categoria,yshift=1cm] {Nome} edge (categoria);
            % relazione Copia Libro - Libro
            \node[relationship] (e_copia_di) [above of=libro] {È copia di} edge node[right]{(0,N)} (libro);
            % entità Copia libro
            \node[entity] (copia_libro) [above of=e_copia_di] {Copia libro} edge node[right,yshift=-15]{(1,1)} (e_copia_di);
            % attributi Copia libro
            \node[attribute] (data_aggiunta) [left of=copia_libro] {Data aggiunta} edge (copia_libro);
            % relazione Prestito
            \node[relationship] (prestito) [above of=copia_libro] {Prestito} edge node[right]{(0,1)} (copia_libro);
            % attributi Prestito
            \node[attribute] (data_inizio) [above left of=prestito,yshift=-1.5cm,xshift=-2cm] {Data inizio} edge (prestito);
            \node[attribute] (data_restituzione) [left of=prestito,xshift=-1cm] {Data restituzione} edge (prestito);
            \node[attribute] (data_fine) [below left of=prestito,yshift=1.5cm,xshift=-2cm] {Data fine} edge (prestito);
            % entità Membro
            \node[entity] (membro) [above of=prestito] {Membro} edge node[right]{(0,N)} (prestito);
            % attributi Membro
            \node[attribute] (ammonizioni) [below of=membro,yshift=1.5cm,xshift=3cm] {Ammonizioni} edge (membro);
            \node[attribute] (nome_membro) [left of=membro] {Nome} edge (membro);
            \node[attribute] (cognome_membro) [above left of=membro,yshift=-1cm] {Cognome} edge (membro);
            \node[attribute] (stato_iscrizione) [above of=membro,yshift=-2cm,xshift=2cm] {Stato iscr.} edge (membro);
            \node[attribute] (datanascita_membro) [above of=membro,yshift=-1cm] {Data di nascita} edge (membro);
            % relazione Copia libro - Biblioteca
            \node[relationship] (presenza) [right of=copia_libro] {Presenza} edge node[below,yshift=-4,xshift=10]{(1,1)} (copia_libro);
            % entità Biblioteca
            \node[entity] (biblioteca) [right of=presenza] {Biblioteca} edge node[above]{(0,N)} (presenza);
            % attributi Biblioteca
            \node[attribute] (nome_biblioteca) [below of=biblioteca] {Nome} edge (biblioteca);
            % relazione Membro - Indirizzo
            \node[relationship] (residenza) [right of=membro] {Residenza} edge node[above]{(1,1)} (membro);
            % entità Indirizzo
            \node[entity] (indirizzo) [right of=residenza] {Indirizzo} edge node[above]{(0,N)} (residenza);
            % attributi Indirizzo
            \node[attribute] (nome_via) [above left of=indirizzo,xshift=1cm] {Nome via} edge (indirizzo);
            \node[attribute] (numero_civico) [above of=indirizzo] {N. civico} edge (indirizzo);
            \node[attribute] (interno) [above right of=indirizzo,xshift=-1cm] {Interno} edge[densely dotted] (indirizzo);
            % relazione Indirizzo - Comune
            \node[relationship] (ind_appartenenza) [right of=indirizzo] {\small Appartenenza} edge node[above,xshift=2]{(1,1)} (indirizzo);
            % entità Comune
            \node[entity] (comune) [right of=ind_appartenenza] {Comune} edge node[above,xshift=-2]{(0,N)} (ind_appartenenza);
            % attributi Comune
            \node[attribute] (nome_comune) [above of=comune] {Nome} edge (comune);
            % relazione Comune - Provincia
            \node[relationship] (com_appartenenza) [right of=comune] {\small Appartenenza} edge node[above,xshift=2]{(1,1)} (comune);
            % entità Provincia
            \node[entity] (provincia) [right of=com_appartenenza] {Provincia} edge node[above,xshift=-2]{(1,N)} (com_appartenenza);
            % attributi Provincia
            \node[attribute] (nome_prov) [above of=provincia] {Nome} edge (provincia);
            % relazione Biblioteca - Gruppo
            \node[relationship] (bib_appartiene) [right of=biblioteca] {\small Appartenenza} edge node[above,xshift=2]{(0,N)} (biblioteca);
            % entità Gruppo
            \node[entity] (gruppo) [right of=bib_appartiene] {Gruppo} edge node[above,xshift=-2]{(0,N)} (bib_appartiene);
            % attributi Gruppo
            \node[attribute] (nome_gruppo) [right of=gruppo] {Nome} edge (gruppo);
            % relazione Biblioteca - Indirizzo
            \node[relationship] (Ubicazione) [above of=biblioteca] {Ubicazione} edge (biblioteca) edge (indirizzo);
        \end{tikzpicture}
    \end{center}
\end{frame}

\begin{frame}{Vincoli non esprimibili}
    \begin{itemize}
        \item I membri devono essere maggiorenni
        \item I membri possono prendere in prestito al massimo 5 libri alla volta
        \item La data di restituzione deve essere precedente alla data di fine prestito per evitare l'ammonizione
        \item Dopo 3 ammonizioni, l'iscrizione viene sospesa per 30 giorni durante i quali non è possibile prendere in prestito libri
        \item Il valore di ``Stato iscrizione'' può essere solo ``Attiva'' e ``Sospesa''
    \end{itemize}
\end{frame}

\begin{frame}{Progettazione logica}
    \begin{itemize}
        \item Analisi delle prestazioni
            \begin{itemize}
                \item Tavola dei volumi
                \item Tabella delle operazioni
            \end{itemize}
        \item Analisi delle ridondanze
        \item Partizionamento/Accorpamento di entità e relazioni
        \item Scelta identificatori principali
        \item Modello ER ristrutturato
        \item Schema logico
    \end{itemize}
\end{frame}

\begin{frame}{Analisi delle prestazoini - Tavola dei volumi}
    \scriptsize
    \def\arraystretch{1.5}
    \begin{center}
        \begin{tabular}{|c|m{2.7cm}|m{5.5cm}|}
            \hline
            \textbf{Elemento}   & \textbf{Volume} & \textbf{Commento} \\ \hline
            Biblioteca          & 264 & Numero attuale preso dal sito ufficale \\ \hline
            Gruppo              & 15 & Numero attuale preso dal sito ufficale \\ \hline
            Membro              & 250000 & Stimato come il 20\% della popolazione \\ \hline
            Copia libro         & 5544000 & Stimato come patrimonio librario medio in Friuli (21247) per il numero di biblioteche \\ \hline
            Libro               & 1848000 & Stima di 3 copie per libro \\ \hline
            Comune              & 216 & Numero esatto \\ \hline
            Provincia           & 4 & Numero esatto \\ \hline
            Categoria           & 100 & Stima \\ \hline
            Autori              & 616000 & Stima di 3 libri per autore \\ \hline
            Prestito            & 2208000 & Stimato come numero di prestiti nel 2012 in Friuli (8366) per il numero di biblioteche \\ \hline
        \end{tabular}
    \end{center}
\end{frame}

\begin{frame}{Analisi delle prestazioni - Tabella delle operazioni}
    \begin{columns}[c]
        \begin{column}[T]{.45\textwidth}
            \begin{itemize}
                \item Molto frequenti
                    \begin{itemize}
                        \item Ricerca dei libri dato
                            \begin{itemize}
                                \item il codice ISBN
                                \item e/o uno specifico autore
                                \item e/o una specifica categoria
                                \item e/o l'anno di pubblicazione
                                \item e/o una stringa che deve essere contenuta nel titolo
                            \end{itemize}
                            in
                            \begin{itemize}
                                \item tutte le biblioteche della regione
                                \item o in un gruppo di biblioteche
                                \item o in una specifica biblioteca
                            \end{itemize}
                    \end{itemize}
                \item Poco frequenti
                    \begin{itemize}
                        \item Iscrizione nuovo membro
                        \item Inserimento nuova copia in catalogo
                    \end{itemize}
            \end{itemize}
        \end{column}
        \begin{column}[T]{.45\textwidth}
            \begin{itemize}
                \item Frequenti
                    \begin{itemize}
                        \item Prestito libro
                            \begin{itemize}
                                \item inizio
                                \item termine
                            \end{itemize}
                        \item Inserimento nuova copia in catalogo
                            \begin{itemize}
                                \item eventuale creazione \textit{libro} se non presente
                            \end{itemize}
                    \end{itemize}
                \item Molto poco frequenti
                    \begin{itemize}
                        \item Creazione nuova biblioteca
                            \begin{itemize}
                                \item creazione indirizzo biblioteca
                            \end{itemize}
                        \item Creazione nuova categoria
                    \end{itemize}
            \end{itemize}
        \end{column}
    \end{columns}
\end{frame}

\begin{frame}{Analisi delle ridondanze}
    È presente un solo ciclo, ma non è una ridondanza perchè gli indirizzi possono essere di un membro o di una biblioteca, e non si può ricavare l'uno dall'altro.\\

    Non sono presenti altre ridondanze.
\end{frame}

\begin{frame}{Partizionamento/Accorpamento di entità e rel.}
    Non sono presenti attributi composti nè multivalore.\\

    Si potrebbe salvare l'entità Indirizzo nella forma di attributi di Membro e Biblioteca rispettivamente, ma è stato scelto fin dall'inizio di mantenere un'entità separata per evitare ridondanze, dal momento che più membri possono risiedere presso lo stesso indirizzo. 
\end{frame}

\begin{frame}{Scelta identificatori principali}
    \begin{itemize}
        \item Membro: ID membro (aggiunto)
        \item Indirizzo: ID indirizzo (aggiunto)
        \item Comune: codice catastale
        \item Provincia: sigla
        \item Biblioteca: ID biblioteca (aggiunto)
        \item Gruppo: ID gruppo (aggiunto)
        \item Libro: codice ISBN
        \item Autore: ID autore (aggiunto)
        \item Categoria: ID categoria (aggiunto)
        \item Copia libro: ID copia (aggiunto), ISBN libro (esterno), ID biblioteca (esterno)
    \end{itemize}
\end{frame}

\begin{frame}{Modello ER ristrutturato}
    \begin{center}
        \begin{tikzpicture}[node distance=9em, every node/.style={scale=0.3}]
            % entità Libro
            \node[entity] (libro) {Libro};
            % attributi Libro
            \node[attribute] (nome_libro) [below of=libro] {Nome} edge (libro);
            \node[attribute] (isbn) [below left of=libro,xshift=1cm] {\ul{ISBN}} edge (libro);
            \node[attribute] (anno) [below right of=libro,xshift=-1cm] {Anno} edge (libro);
            % relazione Libro - Autore
            \node[relationship, xshift=1cm] (scrittura) [right of=libro] {Scrittura} edge node[above]{(1,N)} (libro);
            % entità Autore
            \node[entity] (autore) [right of=scrittura] {Autore} edge node[above,xshift=-1]{(0,N)} (scrittura);
            % attributi Autore
            \node[attribute] (id_autore) [below right of=autore,yshift=1cm] {\ul{ID}} edge (autore);
            \node[attribute] (nome_autore) [right of=autore] {Nome} edge (autore);
            \node[attribute] (cognome_autore) [above right of=autore,yshift=-1cm] {Cognome} edge (autore);
            % relazione Libro - Categoria
            \node[relationship, xshift=-1cm] (libro_appartenenza) [left of=libro] {\small Appartenenza} edge node[above]{(1,N)} (libro);
            % entità Categoria
            \node[entity] (categoria) [left of=libro_appartenenza] {Categoria} edge node[above,xshift=2]{(0,N)} (libro_appartenenza);
            % attributi Categoria
            \node[attribute] (nome_categoria) [above left of=categoria,xshift=1cm] {Nome} edge (categoria);
            \node[attribute] (id_categoria) [above of=categoria] {\ul{ID}} edge (categoria);
            % relazione Copia Libro - Libro
            \node[relationship] (e_copia_di) [above of=libro] {È copia di} edge node[right]{(0,N)} (libro);
            % entità Copia libro
            \node[entity] (copia_libro) [above of=e_copia_di] {Copia libro} edge node[right,yshift=-15]{(1,1)} (e_copia_di);
            % attributi Copia libro
            \node[attribute] (id_copia) [below left of=copia_libro,yshift=1cm] {\ul{ID}} edge (copia_libro);
            \node[attribute] (data_aggiunta) [left of=copia_libro] {Data aggiunta} edge (copia_libro);
            % relazione Prestito
            \node[relationship] (prestito) [above of=copia_libro] {Prestito} edge node[right]{(0,1)} (copia_libro);
            % attributi Prestito
            \node[attribute] (data_inizio) [above left of=prestito,yshift=-1.5cm,xshift=-2cm] {Data inizio} edge (prestito);
            \node[attribute] (data_restituzione) [left of=prestito,xshift=-1cm] {Data restituzione} edge (prestito);
            \node[attribute] (data_fine) [below left of=prestito,yshift=1.5cm,xshift=-2cm] {Data fine} edge (prestito);
            % entità Membro
            \node[entity] (membro) [above of=prestito] {Membro} edge node[right]{(0,N)} (prestito);
            % attributi Membro
            \node[attribute] (id_membro) [below left of=membro,yshift=1cm] {\ul{ID}} edge (membro);
            \node[attribute] (ammonizioni) [below of=membro,yshift=1.5cm,xshift=3cm] {Ammonizioni} edge (membro);
            \node[attribute] (nome_membro) [left of=membro] {Nome} edge (membro);
            \node[attribute] (cognome_membro) [above left of=membro,yshift=-1cm] {Cognome} edge (membro);
            \node[attribute] (stato_iscrizione) [above of=membro,yshift=-2cm,xshift=2cm] {Stato iscr.} edge (membro);
            \node[attribute] (datanascita_membro) [above of=membro,yshift=-1cm] {Data di nascita} edge (membro);
            % relazione Copia libro - Biblioteca
            \node[relationship] (presenza) [right of=copia_libro] {Presenza} edge node[below,yshift=-4,xshift=10]{(1,1)} (copia_libro);
            % entità Biblioteca
            \node[entity] (biblioteca) [right of=presenza] {Biblioteca} edge node[above]{(0,N)} (presenza);
            % attributi Biblioteca
            \node[attribute] (id_biblioteca) [below of=biblioteca] {\ul{ID}} edge (biblioteca);
            \node[attribute] (nome_biblioteca) [below right of=biblioteca,xshift=-1cm] {Nome} edge (biblioteca);
            % relazione Membro - Indirizzo
            \node[relationship] (residenza) [right of=membro] {Residenza} edge node[above]{(1,1)} (membro);
            % entità Indirizzo
            \node[entity] (indirizzo) [right of=residenza] {Indirizzo} edge node[above]{(0,N)} (residenza);
            % attributi Indirizzo
            \node[attribute] (id_indirizzo) [above left of=indirizzo,yshift=-1cm] {\ul{ID}} edge (indirizzo);
            \node[attribute] (nome_via) [above left of=indirizzo,xshift=1cm] {Nome via} edge (indirizzo);
            \node[attribute] (numero_civico) [above of=indirizzo] {N. civico} edge (indirizzo);
            \node[attribute] (interno) [above right of=indirizzo,xshift=-1cm] {Interno} edge[densely dotted] (indirizzo);
            % relazione Indirizzo - Comune
            \node[relationship] (ind_appartenenza) [right of=indirizzo] {\small Appartenenza} edge node[above,xshift=2]{(1,1)} (indirizzo);
            % entità Comune
            \node[entity] (comune) [right of=ind_appartenenza] {Comune} edge node[above,xshift=-2]{(0,N)} (ind_appartenenza);
            % attributi Comune
            \node[attribute] (cod_catastale) [above of=comune] {\ul{Codice catastale}} edge (comune);
            \node[attribute] (nome_comune) [above right of=comune,xshift=-1cm] {Nome} edge (comune);
            % relazione Comune - Provincia
            \node[relationship] (com_appartenenza) [right of=comune] {\small Appartenenza} edge node[above,xshift=2]{(1,1)} (comune);
            % entità Provincia
            \node[entity] (provincia) [right of=com_appartenenza] {Provincia} edge node[above,xshift=-2]{(1,N)} (com_appartenenza);
            % attributi Provincia
            \node[attribute] (sigla_prov) [above of=provincia] {\ul{Sigla}} edge (provincia);
            \node[attribute] (nome_prov) [above right of=provincia,xshift=-1cm] {Nome} edge (provincia);
            % relazione Biblioteca - Gruppo
            \node[relationship] (bib_appartiene) [right of=biblioteca] {\small Appartenenza} edge node[above,xshift=2]{(0,N)} (biblioteca);
            % entità Gruppo
            \node[entity] (gruppo) [right of=bib_appartiene] {Gruppo} edge node[above,xshift=-2]{(0,N)} (bib_appartiene);
            % attributi Gruppo
            \node[attribute] (nome_gruppo) [above right of=gruppo,xshift=-1cm] {Nome} edge (gruppo);
            \node[attribute] (id_gruppo) [above of=gruppo] {\ul{ID}} edge (gruppo);
            % relazione Biblioteca - Indirizzo
            \node[relationship] (Ubicazione) [above of=biblioteca] {Ubicazione} edge (biblioteca) edge (indirizzo);

            % arco identificatore esterno copia libro
            \centerarc[-*](0,2.1)(-160:25:0.5);
        \end{tikzpicture}
    \end{center}
\end{frame}

\begin{frame}{Schema logico}
    \scriptsize
    \begin{itemize}
        \item Membro (\ul{ID}\footnote{Le PK sono sottolineate, le FK sono in corsivo.}, \textit{IDindirizzo}, Nome, Cognome, DataDiNascita, StatoIscrizione, Ammonizioni)
        \item Indirizzo (\ul{ID}, \textit{CodiceCatastaleComune}, NomeVia, Ncivico, Interno)
        \item Comune (\ul{CodiceCatastale}, \textit{SiglaProvincia}, Nome)
        \item Provincia (\ul{Sigla}, Nome)
        \item Biblioteca (\ul{ID}, \textit{IDindirizzo}, Nome)
        \item BibliotecaGruppo (\ul{\textit{IDbiblioteca}}, \ul{\textit{IDgruppo}})
        \item Gruppo (\ul{ID}, Nome)
        \item Libro (\ul{ISBN}, Nome)
        \item LibroAutore (\ul{\textit{ISBN}}, \ul{\textit{IDautore}})
        \item Autore (\ul{ID}, Nome, Cognome)
        \item LibroCategoria (\ul{\textit{ISBN}}, \ul{\textit{IDcategoria}})
        \item Categoria (\ul{ID}, Nome)
        \item CopiaLibro (\ul{ID}, \ul{\textit{ISBNlibro}}, \ul{\textit{IDbiblioteca}}, DataAggiunta)
        \item Prestito (\ul{ID}, \textit{IDcopia}, \textit{IDmembro}, DataInizio, DataFine, DataRestituzione)
    \end{itemize}
\end{frame}

\begin{frame}{Progettazione fisica}
    \begin{itemize}
        \item Scelta degli indici
        \item Definizione dati
        \item Definizione trigger
        \item Definizione viste
        \item User defined functions
    \end{itemize}
\end{frame}

\begin{frame}{Progettazione fisica - Scelta degli indici}
    È buona norma definire gli indici sulle colonne che corrispondono a chiavi esterne, ma InnoDB li definisce in automatico.\\

    Dalla versione 5.6 di MySQL, questo motore supporta gli indici di tipo FULLTEXT. Scegliamo quindi di definire questo indice sulla colonna \textit{nome} della tabella \textit{libro} data la frequenza e la tipologia delle ricerche che la coinvolgono.\\

    Aggiungiamo inoltre un indice sulla colonna \textit{anno} della stessa tabella.
\end{frame}

\begin{frame}[fragile]{Progettazione fisica - Definizione dati}
    \begin{minted}[fontsize=\scriptsize]{mysql}
-- Creazione database
CREATE DATABASE `bibfvg`
    CHARACTER SET utf8 -- Usiamo UTF-8 come codifica
    COLLATE utf8_general_ci;

-- Definizione database predefinito
USE `bibfvg`;

-- Usiamo InnoDB (che supporta le FK) come default engine
-- per la creazione di tabelle nella sessione corrente
SET storage_engine=InnoDB;
    \end{minted}
\end{frame}

\begin{frame}[fragile]{Progettazione fisica - Definizione dati}
    \begin{columns}[T]
        \begin{column}[T]{.45\textwidth}
            \begin{minted}[fontsize=\tiny]{mysql}
-- Creazione tabella ‘provincia‘
CREATE TABLE `provincia` (
    -- La sigla e‘ sempre lunga 2
    `sigla` CHAR(2) NOT NULL,
    `nome` VARCHAR(40) NOT NULL,

    PRIMARY KEY (`sigla`)
);
            \end{minted}
        \end{column}
        \begin{column}[T]{.45\textwidth}
            \begin{minted}[fontsize=\tiny]{mysql}
-- Creazione tabella ‘comune‘
CREATE TABLE `comune` (
    -- Il cod. cat. e‘ sempre lungo 4
    `codice_catastale` CHAR(4) NOT NULL,
    `nome` VARCHAR(40) NOT NULL,
    `sigla_prov` CHAR(2) NOT NULL,

    PRIMARY KEY (`codice_catastale`),

    FOREIGN KEY (`sigla_prov`)
        REFERENCES `provincia`(`sigla`)
        ON DELETE CASCADE
        ON UPDATE CASCADE
);
            \end{minted}
        \end{column}
    \end{columns}
\end{frame}

\begin{frame}[fragile]{Progettazione fisica - Definizione dati}
    \begin{columns}[T]
        \begin{column}[T]{.45\textwidth}
            \begin{minted}[fontsize=\tiny]{mysql}
-- Creazione tabella ‘indirizzo‘
CREATE TABLE `indirizzo` (
    `id` INT NOT NULL AUTO_INCREMENT,
    `codice_catastale_com` CHAR(4) NOT NULL,
    -- N. civico e interno possono
    -- contenere lettere
    `nome_via` VARCHAR(40) NOT NULL,
    `ncivico` VARCHAR(5) NOT NULL,
    `interno` VARCHAR(5) DEFAULT NULL,

    PRIMARY KEY (`id`),

    FOREIGN KEY (`codice_catastale_com`)
        REFERENCES `comune`(`codice_catastale`)
        ON DELETE CASCADE
        ON UPDATE CASCADE
);
            \end{minted}
        \end{column}
        \begin{column}[T]{.45\textwidth}
            \begin{minted}[fontsize=\tiny]{mysql}
-- Creazione tabella ‘membro‘
CREATE TABLE `membro` (
    `id` INT NOT NULL AUTO_INCREMENT,
    `id_indirizzo` INT NOT NULL,
    `nome` VARCHAR(40) NOT NULL,
    `cognome` VARCHAR(40) NOT NULL,
    `data_di_nascita` DATE NOT NULL,
    -- Lo stato puo‘ assumere solo 2 valori
    `stato_iscrizione` ENUM('attiva',
                            'sospesa'),
    `ammonizioni` INT NOT NULL,

    PRIMARY KEY (`id`),

    FOREIGN KEY (`id_indirizzo`)
        REFERENCES `indirizzo`(`id`)
        ON DELETE CASCADE
        ON UPDATE CASCADE
);
            \end{minted}
        \end{column}
    \end{columns}
\end{frame}

\begin{frame}[fragile]{Progettazione fisica - Definizione dati}
    \begin{columns}[T]
        \begin{column}[T]{.45\textwidth}
            \begin{minted}[fontsize=\tiny]{mysql}
-- Creazione tabella ‘gruppo‘
CREATE TABLE `gruppo` (
    `id` INT NOT NULL AUTO_INCREMENT,
    `nome` VARCHAR(40) NOT NULL,

    PRIMARY KEY (`id`)
);

-- Creazione tabella ‘biblioteca‘
CREATE TABLE `biblioteca` (
    `id` INT NOT NULL AUTO_INCREMENT,
    `id_indirizzo` INT NOT NULL,
    `nome` VARCHAR(40) NOT NULL,

    PRIMARY KEY (`id`),

    FOREIGN KEY (`id_indirizzo`)
        REFERENCES `indirizzo`(`id`)
        ON DELETE CASCADE
        ON UPDATE CASCADE
);
            \end{minted}
        \end{column}
        \begin{column}[T]{.45\textwidth}
            \begin{minted}[fontsize=\tiny]{mysql}
-- Creazione tabella ‘biblioteca_gruppo‘
CREATE TABLE `biblioteca_gruppo` (
    `id_biblioteca` INT NOT NULL,
    `id_gruppo` INT NOT NULL,

    PRIMARY KEY (`id_biblioteca`, `id_gruppo`),
    
    FOREIGN KEY (`id_biblioteca`)
        REFERENCES `biblioteca`(`id`)
        ON DELETE CASCADE
        ON UPDATE CASCADE,

    FOREIGN KEY (`id_gruppo`)
        REFERENCES `gruppo`(`id`)
        ON DELETE CASCADE
        ON UPDATE CASCADE
);
            \end{minted}
        \end{column}
    \end{columns}
\end{frame}

\begin{frame}[fragile]{Progettazione fisica - Definizione dati}
    \begin{columns}[T]
        \begin{column}[T]{.45\textwidth}
            \begin{minted}[fontsize=\tiny]{mysql}
-- Creazione tabella ‘libro‘
CREATE TABLE `libro` (
    -- Usiamo l’ISBN13 che e‘ sempre lungo 13
    `isbn` CHAR(13) NOT NULL,
    `nome` VARCHAR(80) NOT NULL,
    `anno` YEAR(4) NOT NULL,

    PRIMARY KEY (`isbn`),
    FULLTEXT (`nome`),
    INDEX (`anno`)
);
            \end{minted}
        \end{column}
        \begin{column}[T]{.45\textwidth}
            \begin{minted}[fontsize=\tiny]{mysql}
-- Creazione tabella ‘autore‘
CREATE TABLE `autore` (
    `id` INT NOT NULL AUTO_INCREMENT,
    `nome` VARCHAR(40) NOT NULL,
    `cognome` VARCHAR(40) NOT NULL,

    PRIMARY KEY (`id`)
);

-- Creazione tabella ‘categoria‘
CREATE TABLE `categoria` (
    `id` INT NOT NULL AUTO_INCREMENT,
    `nome` VARCHAR(40),

    PRIMARY KEY (`id`)
);
            \end{minted}
        \end{column}
    \end{columns}
\end{frame}

\begin{frame}[fragile]{Progettazione fisica - Definizione dati}
    \begin{columns}[T]
        \begin{column}[T]{.45\textwidth}
            \begin{minted}[fontsize=\tiny]{mysql}
-- Creazione tabella ‘libro_autore‘
CREATE TABLE `libro_autore` (
    `isbn_libro` CHAR(13) NOT NULL,
    `id_autore` INT NOT NULL,

    PRIMARY KEY (`isbn_libro`, `id_autore`),
    
    FOREIGN KEY (`isbn_libro`)
        REFERENCES `libro`(`isbn`)
        ON DELETE CASCADE
        ON UPDATE CASCADE,

    FOREIGN KEY (`id_autore`)
        REFERENCES `gruppo`(`id`)
        ON DELETE CASCADE
        ON UPDATE CASCADE
);
            \end{minted}
        \end{column}
        \begin{column}[T]{.45\textwidth}
            \begin{minted}[fontsize=\tiny]{mysql}
-- Creazione tabella ‘libro_categoria‘
CREATE TABLE `libro_categoria` (
    `isbn_libro` CHAR(13) NOT NULL,
    `id_categoria` INT NOT NULL,

    PRIMARY KEY (`isbn_libro`, `id_categoria`),
    
    FOREIGN KEY (`isbn_libro`)
        REFERENCES `libro`(`isbn`)
        ON DELETE CASCADE
        ON UPDATE CASCADE,

    FOREIGN KEY (`id_categoria`)
        REFERENCES `gruppo`(`id`)
        ON DELETE CASCADE
        ON UPDATE CASCADE
);
            \end{minted}
        \end{column}
    \end{columns}
\end{frame}

\begin{frame}[fragile]{Progettazione fisica - Definizione dati}
    \begin{columns}[T]
        \begin{column}[T]{.45\textwidth}
            \begin{minted}[fontsize=\tiny]{mysql}
-- Creazione tabella ‘copia_libro‘
CREATE TABLE `copia_libro` (
    `id` INT NOT NULL AUTO_INCREMENT,
    `isbn_libro` CHAR(13) NOT NULL,
    `id_biblioteca` INT NOT NULL,

    PRIMARY KEY (`id`,
                 `isbn_libro`,
                 `id_biblioteca`),

    FOREIGN KEY (`isbn_libro`)
        REFERENCES `libro`(`isbn`)
        ON DELETE CASCADE
        ON UPDATE CASCADE,

    FOREIGN KEY (`id_biblioteca`)
        REFERENCES `biblioteca`(`id`)
        ON DELETE CASCADE
        ON UPDATE CASCADE
);
            \end{minted}
        \end{column}
        \begin{column}[T]{.45\textwidth}
            \begin{minted}[fontsize=\tiny]{mysql}
-- Creazione tabella ‘prestito‘
CREATE TABLE `prestito` (
    `id` INT NOT NULL AUTO_INCREMENT,
    `id_copia` INT NOT NULL,
    `id_membro` INT NOT NULL,
    `data_inizio` DATE NOT NULL,
    `data_fine` DATE DEFAULT NULL,
    `data_restituzione` DATE DEFAULT NULL,

    PRIMARY KEY (`id`),

    FOREIGN KEY (`id_copia`)
        REFERENCES `copia_libro`(`id`)
        ON DELETE CASCADE
        ON UPDATE CASCADE,

    FOREIGN KEY (`id_membro`)
        REFERENCES `membro`(`id`)
        ON DELETE CASCADE
        ON UPDATE CASCADE
);
            \end{minted}
        \end{column}
    \end{columns}
\end{frame}

\begin{frame}[fragile]{Progettazione fisica - Definizione trigger}
    \begin{minted}[fontsize=\tiny]{mysql}
DELIMITER //
-- Trigger per la verifica della possibilità di richiedere un prestito
CREATE TRIGGER check_prestito BEFORE INSERT ON `prestito`
    FOR EACH ROW
    BEGIN
        DECLARE NUMPRESTITI INT;
        DECLARE STATOISCRIZIONE VARCHAR(7);

        SET STATOISCRIZIONE = (SELECT `stato_iscrizione` FROM `membro` WHERE `id` = NEW.id_membro);
        IF STATOISCRIZIONE = 'Sospesa' THEN -- Se l'iscrizione è sospesa
            IF (SELECT TIMESTAMPDIFF(DAY, `data_restituzione`, CURDATE()) FROM `prestito`
                WHERE `id` = NEW.id_membro ORDER BY `data_restituzione` DESC LIMIT 1) > 30 THEN
                -- Se sono passati > 30gg dall'ultima restituzione, riattivo l'iscrizione
                UPDATE `membro`
                SET `stato_iscrizione` = 'Attiva';
            ELSE -- Altrimenti impedisco l'operazione
                SIGNAL SQLSTATE '45000' SET MESSAGE_TEXT = "L'iscrizione e` sospesa";
            END IF;
        END IF;
        
        SET NUMPRESTITI = (SELECT COUNT(*) FROM `prestito` WHERE `id_membro` = NEW.id_membro
                           AND `data_fine` IS NULL);
        IF NUMPRESTITI > 4 THEN -- Se il membro ha già preso in prestito 5 libri, impedisco l'operazione
            SIGNAL SQLSTATE '45000' SET MESSAGE_TEXT = 'Si possono prendere in prestito'
                                                       ' al massimo 5 libri alla volta';
        END IF;
    END //
DELIMITER ;
    \end{minted}
\end{frame}

\begin{frame}[fragile]{Progettazione fisica - Definizione trigger}
    \begin{minted}[fontsize=\tiny]{mysql}
DELIMITER //
-- Trigger per la segnalazione di ritardi/ammonizioni
CREATE TRIGGER check_ritardo_restituzione BEFORE UPDATE ON `prestito`
    FOR EACH ROW
    BEGIN
        IF NEW.data_restituzione > OLD.data_fine THEN -- Se la restituzione è in ritardo
            IF (SELECT `ammonizioni` FROM `membro` WHERE `id_membro` = NEW.id_membro) > 2 THEN
                -- Se il membro ha già almeno 3 ammonizioni sospendo l'iscrizione, azzerandole
                UPDATE `membro` SET `ammonizioni` = 0, `stato_iscrizione` = 'Sospesa'
                WHERE id = NEW.id_membro;
            ELSE -- Altrimenti incremento il numero di ammonizioni
                UPDATE `membro` SET `ammonizioni` = `ammonizioni` + 1
                WHERE id = NEW.id_membro;
            END IF;
        END IF;
    END //

-- Trigger per permettere solo ai maggiorenni di iscriversi
CREATE TRIGGER check_maggiorenne BEFORE INSERT ON `membro`
    FOR EACH ROW
    BEGIN
        IF TIMESTAMPDIFF(YEAR, NEW.data_di_nascita, CURDATE()) < 18 THEN
            -- Se il membro non è maggiorenne, impedisco l'iscrizione
            SIGNAL SQLSTATE '45000' SET MESSAGE_TEXT = 'I membri devono essere maggiorenni';
        END IF;
    END //
DELIMITER ;
    \end{minted}
\end{frame}

\begin{frame}[fragile]{Progettazione fisica - Definizione viste}
    \begin{minted}[fontsize=\scriptsize]{mysql}
-- Creiamo la vista contenente i 10 libri più noleggiati
CREATE VIEW `top10_libri` AS
    SELECT `l`.`nome` AS `nome_libro`,
           `l`.`isbn` AS `isbn`,
           COUNT(*) AS `numero_prestiti`
    FROM `prestito` AS `p`
        INNER JOIN `copia_libro` AS `c` ON `p`.`id_copia` = `c`.`id` 
        INNER JOIN `libro` AS `l` ON `c`.`isbn_libro` = `l`.`isbn`
    WHERE `data_restituzione` IS NOT NULL
    GROUP BY `isbn`
    ORDER BY `numero_prestiti` DESC
    LIMIT 10;

-- Diamo all'utente esterno la possibilità di leggere la vista
-- contenente la classifica
GRANT SELECT ON `bibfvg`.`top10_libri`
TO `utente_esterno`@`localhost`;
    \end{minted}
\end{frame}

\begin{frame}[fragile]{Progettazione fisica - User defined functions}
    \begin{minted}[fontsize=\tiny]{mysql}
DELIMITER //
-- Funzione per la conversione dell’ISBN10 a ISBN13
CREATE FUNCTION ISBN10TO13 (ISBN10 CHAR(10))
    RETURNS CHAR(13)
    DETERMINISTIC
    BEGIN
        DECLARE ISBN13 CHAR(13);
        DECLARE CHECKSUM, I INT;

        SET ISBN13  = CONCAT('978' , LEFT(ISBN10, 9));

        SET I = 1, CHECKSUM = 0;
        WHILE I < 12 DO
            -- Sommo al checksum le cifre dispari, e quelle pari moltiplicate per 3
            SET CHECKSUM = CHECKSUM
                           + SUBSTRING(ISBN13, I, 1)
                           + SUBSTRING(ISBN13, I+1, 1) * 3;
            SET I = I + 2;
        END WHILE;

        SET CHECKSUM = (10 - (CHECKSUM % 10)) % 10;

        -- ISBN13 = ’978’ + prime 9 cifre dell’ISBN10 + checksum digit
        RETURN CONCAT(ISBN13, CONVERT(CHECKSUM, CHAR(1)));
    END //
DELIMITER ;
    \end{minted}
\end{frame}

\begin{frame}[fragile]{Progettazione fisica - User defined functions}
    \begin{minted}[fontsize=\tiny]{mysql}
DELIMITER //
-- Funzione per la conversione dell’ISBN13 a ISBN10
CREATE FUNCTION ISBN13TO10 (ISBN13 CHAR(13))
    RETURNS CHAR(10)
    DETERMINISTIC
    BEGIN
        DECLARE ISBN10 CHAR(10);
        DECLARE CHECKSUM, I INT;

        IF LEFT(ISBN13, 3) <> '978' THEN -- Ha senso convertire solo gli ISBN13 che iniziano con '978'
            SIGNAL SQLSTATE '45000' SET MESSAGE_TEXT = 'ISBN13 non convertibile';
        END IF;

        SET ISBN10  = SUBSTRING(ISBN13, 4, 9), I = 1, CHECKSUM = 0;
        WHILE I < 10 DO
            -- Sommo al checksum le cifre dispari, e quelle pari moltiplicate per 3
            SET CHECKSUM = CHECKSUM + SUBSTRING(ISBN10, I, 1) * (11-I), I = I + 1;
        END WHILE;

        SET CHECKSUM = (11 - (CHECKSUM % 11)) % 11;
        -- ISBN10 = ISBN13 senza le prime 3 e l'ultima cifra + checksum digit
        IF CHECKSUM = 10 THEN
            RETURN CONCAT(ISBN10, 'X');
        ELSE
            RETURN CONCAT(ISBN10, CONVERT(CHECKSUM, CHAR(1));
        END IF;
    END //
DELIMITER ;
    \end{minted}
\end{frame}